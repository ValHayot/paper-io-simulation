\documentclass[conference]{IEEEtran}

\begin{document}
\title{I/O simulation extension for SimGrid framework}
\author{Hoang-Dung Do}
\maketitle

	\begin{abstract}
		\begin{itemize}
			\item The I/O bottleneck in HPC and the need of experiments.
			\item HPC experiment frameworks and advantages of SimGrid.
			\item The missing of the ability to simulate page cache, the goal of this paper.
			\item Principle of the simulator, experiment scenarios and comparisons.
			\item Brief discussion on results and future work.
		\end{itemize}
	\end{abstract}

	\section{Introduction}
		\begin{itemize}
			\item HPC, the bottleneck in I/O and the demand of HPC experiments. 
			\item Difficulties in conducting high performance computing experiments and the need of simulation frameworks.
			\item Existing experiment methods, simulators, simulation frameworks. The advantages of SimGrid compared to others \cite{casanova2008, lebre2015}. The missing of the ability to simulate page cache in SimGrid \cite{lebre2015}.
			\item The objective of the paper: Add capability to simulate I/O with page cache in SimGrid.
		\end{itemize}
	\section{Related Work}			
		
		\subsection{Page cache}
			\begin{itemize}
				\item What is page cache? How it works \cite{linuxdev3rd2010}. Effects and importance of page cache.
				\item Introduce some existing strategies with some highlighted pros and cons.
				\item Current implementation in Linux and some reasons why it is chosen to be implemented (implementation complexity, effectiveness, overhead, etc) \cite{linuxdev3rd2010}
			\end{itemize}									

		\subsection{Simulators}
			\begin{itemize}
				\item Discuss some existing methods, simulation frameworks to conduct HPC experiments. Compare pros and cons (accuracy, simulation time, usability) of some simulators (SimGrid, GridSim).
				\item Discuss the pros of SimGrid and the reasons why we chose it to extend.
			\end{itemize}
			
	\section{Method}

		\subsection{Principle of the simulator}

			\begin{itemize}
				\item Approach: generalize dirty data, dirty ratio, cache eviction strategy implemented in Linux. 
				\item Some implemented details of the simulator.
			\end{itemize}

		\subsection{Implementation}
			\begin{itemize}
				\item Which features of memory are implemented.
				\item Level of granularity, how features are implemented.
				\item Specific implementation details in python and SimGrid.
			\end{itemize}

		\subsection{Experiments}
			Describe data, workflow, number of tasks, task details, environment of each experiment.
	
			\subsubsection{Expriment 1}
				A single pipeline running one node.
			\subsubsection{Expriment 2}
				Multiple pipelines running in parallel on multiple nodes.
			\subsubsection{Expriment 3}
				Same as Experiment 2 but nodes write to a shared file system.
			\subsubsection{Expriment 4}
				A real pipeline (for example a pipeline with nighres)

	\section{Results}
	
		\begin{itemize}

			\item Quantized results: 
				\begin{itemize}
					\item Errors of simulation time and memory used compared to real results.
					\item Simulation time compared to baseline SimGrid.
				\end{itemize} 

			\item Ability of the model to generalize trends seen in memory (amount of dirty data, page cache) and disk throughput.

		\end{itemize}

	\section{Discussion and Future Work}

\bibliographystyle{plain}
\bibliography{citation}

\end{document}